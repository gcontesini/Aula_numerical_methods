\documentclass[twocolumn]{article} \usepackage{graphicx, epsfig} \usepackage{fixltx2e} \usepackage{hyperref} \usepackage{setspace} \usepackage{ctable}

\title{Lab 4\\Computational Physics I - Phys381} \author{Guilherme Contesini, \\ Gerswin Magat , 00325287}

%%%%%%%%%%%%%%%%%%%%%%%%%%%%%%%%%%%%%%%%%%%%%%%%%%%%%%%%%%%%%%%%%%%%%%%%%%%%%%%%%%%%%%%%%%%%%%%%%%%%%

\begin{document} \onecolumn \maketitle \date \begin{abstract} Blah

\end{abstract}

\newpage

%%%%%%%%%%%%%%%%%%%%%%%%%%%%%%%%%%%%%%%%%%%%%%%%%%%%%%%%%%%%%%%%%%%%%%%%%%%%%%%%%%%%%%%%%%%%%%%%%%%%%

\twocolumn

\section{Introduction}

Computers take values by means of binary codes. This means that information is not held in values of base ten, but by values of base 2. A binary number consists only of a 0 or a 1. For example, the integer 5 is represented by the number '101' in binary. to translate a number of base 10 to a binary, we would simply divide by 2 and write down the remainder value in sequence including zeroes. to get a base-10 number from a binary, the opposite procedure is the correct method. Each number in sequence from right to left would correspond to $2^0$, $2^1$, $2^2$, and so on. Taking the previous example, this would look like: \newline

$5_{10} = 101_2 = [(1 \times 2^2) + (0 \times 2^1) + (1 \times 2^0)]_{10}$ \newline $5_{10} = 101_2 = [4 + 0 + 1]_{10}$ \newline $5_{10} = 101_2 = 5_{10}$ \newline

Similar to the method of computing binary vs. base-10 values, computers also interpret values differently than humans. There are methods that computers do in order to contain the information in an ``efficient'' way. What we must keep in mind is that computers allocate a limited amount of memory for any value unless over-ridden by the user. These limitations by the computers can cause issues in terms of precision and accuracy. \newline

The purpose of this lab is to examine the extent at which computers can take information correctly and what distortions could happen if we are not careful with information methods.
\section{Loss of Accuracy}
\subsection{i}
 \onecolumn { \renewcommand{\arraystretch}{2.2} 
 \begin{table}[h!] 
 \begin{tabular}{l c l l l l l} \hline 
 $\epsilon$= & & 0.1 & 0.01 & $1.0 \times 10^{-4}$ & $1.0 \times 10^{-8}$ & $1.0 \times 10^{-16}$ \\ \hline
  $1 - \sqrt{1 - \epsilon}$ & [4-byte] & \underline{0.0513167} & 0.00501257 & 5.00083$\times 10^{-5}$ & 0.00000 & 0.00000 \\ \hline
   $1 - \sqrt{1 - \epsilon}$ & [8-byte] & \underline{0.0513167} & \underline{0.00501256} & \underline{5.00013$\times 10^{-5}$} & \underline{5.00000$\times 10^{-9}$} & 0.00000 \\ \hline
    $\frac{\epsilon}{1 + \sqrt{1-\epsilon}}$ & [4-byte] & \underline{0.0513167} & \underline{0.00501256} & \underline{5.00013$\times 10^{-5}$} & \underline{5.00000$\times 10^{-9}$} & \underline{5.00000$\times 10^{-17}$} \\ \hline $\frac{\epsilon}{2}$ & [4-byte] & 0.0500000 & 0.00500000 & 5.00000$\times 10^{-5}$ & \underline{5.00000$\times 10^{-9}$} & \underline{5.00000$\times 10^{-17}$} \\ \hline
     $\frac{\epsilon}{2}+\frac{\epsilon^2}{8}$ & [4-byte] & 0.0512500 & 0.00501250 & \underline{5.00013$\times 10^{-5}$} & \underline{5.00000$\times 10^{-9}$} & \underline{5.00000$\times 10^{-17}$} \\ \hline
      $\frac{\epsilon}{2}+\frac{\epsilon^2}{8}+\frac{\epsilon^3}{16}$ & [4-byte] & 0.0513125 & \underline{0.00501256} & \underline{5.00013$\times 10^{-5}$} & \underline{5.00000$\times 10^{-9}$} & \underline{5.00000$\times 10^{-17}$} \\ \hline 
      \end{tabular} 
      \caption{Extracted Data with Calculated Values} 
      \end{table} }
\twocolumn

\section{Conclusion}

From the results of the lab, we can gather that accuracy and precision are both vital to receive useable results. The example in class that Dr. Ouyed mentioned was a catastrophic event where American soldiers lost their lives due to the carelessness of the scientists who overlooked the implications of small errors over millions and millions of loops. This shows that these calculations are very important and must be taken seriously, even the error propagations and calculations!

\section{Appendix}

\subsection{Gnuplot Code}

\begin{verbatim} reset

\end{verbatim}

\subsection{Fortran Code}

\begin{verbatim} program main

\end{verbatim}

\end{document}